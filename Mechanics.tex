\section{EVA Software and External Devices [DSPL Template]}
% In this section briefly describe the software and hardware of the robot

\setlength\intextsep{0pt}
%\begin{wrapfigure}[10]{r}{0.3\textwidth}
%	\centering
%	\includegraphics[width=0.4\textwidth]{images/eva.png}
%	\caption{Robot EVA}
%	\label{fig:eva}
%\end{wrapfigure}

\begin{figure}
	\centering
	\includegraphics[width=0.4\textwidth]{images/eva.png}
	\caption{Robot EVA}
	\label{fig:eva}
\end{figure}

We use a standard EVA robot from \textit{Buy'N Large}. No modifications have been applied.

\section*{Robot's Software Description}
% Please describe in this section the software you are using to control your robot. Consider the following example:

\textit{For our robot we are using the following software:}

\begin{itemize}
	\item Platform: \BnL Operating System
	\item Face recognition: None. Not designed for human interaction.
	\item Object recognition: \BnL Green Plant Seeker Algorithm (See previous sections).
	\item Arms control and two-hand coordination: \BnL automatic controller \cite{bnl2}.
\end{itemize}

\section*{External Devices}
% Please describe in this section the external devices used by your robot. Consider the following example:

\textit{EVA robot relies on the following external hardware:}

\begin{itemize}
	\item \BnL Mother-ship
	\item \BnL Data Cluster
	\item $3 \times$ \BnL Ultra-Power laptops.
\end{itemize}

\section*{Cloud Services}
% Please describe in this section the Cloud Services and online software used by your robot. Consider the following example:

\textit{EVA connects the following cloud services:}
\begin{itemize}
	\item Localization and mapping: \BnL Geolocalization system \cite{bnl3}.
	\item Navigation: \BnL Navigator
	\item Speech recognition: \BnL All-purpose recognizer \cite{bnl1}.
	\item Speech generation: \BnL Speech synthesizer.
\end{itemize}
