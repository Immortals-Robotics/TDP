% This paper uses LLNCS macro package for Springer Computer Science proceedings;
% Version 2.20 of 2017/10/04
%
\documentclass[runningheads]{llncs}
%
\usepackage{subcaption}
\captionsetup{compatibility=false}
\usepackage{graphicx}
\usepackage{placeins}
\usepackage{float}
% Used for displaying a sample figure. If possible, figure files should
% be included in EPS format.
%
% If you use the hyperref package, please uncomment the following line
% to display URLs in blue roman font according to Springer's eBook style:
% \renewcommand\UrlFont{\color{blue}\rmfamily}

\begin{document}
%
\title{Immortals 2024 Extended Team Description Paper}
\titlerunning{Immortals 2024 ETDP}

\author{Ali Salehi \and
Mohammad Tabasi \and
Omid Najafi \and
Ali Amoozandeh Nobaveh \and
MohammadHossein Fazeli \and
MohammadAli Ghasemieh \and
MohammadReza Niknezhad \and
Mustafa Talaeezadeh}
%
\authorrunning{Immortals Robotics}
%
\institute{
\url{http://www.immortals-robotics.com}
}
%
\maketitle              % typeset the header of the contribution
%
\begin{abstract}
This paper describes the recent work that has been done by the Immortals Robotics Team for the upcoming RoboCup 2024 competition in Eindhoven, Netherland.

\keywords{RoboCup 2024 \and Small Size League}
\end{abstract}

\section{Introduction}
The Immortals Small Size League team was founded in 2008. We first participated in RoboCup 2009 in Graz. Later, we won several awards. This included 2nd place in RoboCup 2011 in Istanbul, and 3rd place in RoboCup 2023 in Bordeaux. The team currently consists of computer and electrical engineers.

In the past year, our team focused on updating the electronics and software structure \cite{ref_ETDP2023}. These changes were aimed at improving the core functionalities of our system. While these updates represented a major step forward, they also introduced significant instability. Consequently, we were compelled to revert some parts of the system to older designs during RoboCup 2023.

This year, our main focus will shift to enhancing system robustness. We will also capitalize on the improved hardware and software foundation established through modernization efforts. By prioritizing robustness enhancement, we aim to ensure a more stable and reliable performance of our system in future competitions.
 
It is worth mentioning that we will publish our designs and software on our GitHub page \cite{ref_github} shortly after the competition is over.

\begin{figure}
    \centering
    \begin{subfigure}[b]{0.45\textwidth}
         \centering
         \includegraphics[width=\textwidth]{images/std_robot.jpeg}
         \caption{Standard}
         \label{fig:robot_std}
    \end{subfigure}
    \hfill
    \begin{subfigure}[b]{0.5\textwidth}
        \centering
        \includegraphics[width=\textwidth]{images/printed_robot.jpeg}
        \caption{3D-printed}
        \label{fig:robot_printed}
    \end{subfigure}
    \caption{Immortals current robots.}
    \label{fig:robots}
\end{figure}

\section {Mechanics}
The mechanical design team decided not to overhaul the entire system. Instead, they chose to modify design details due to the optimal design of the previous robots. This change aims to improve the manufacturability and durability of the parts. Recent advances in additive manufacturing techniques have made 3D-printed parts more accurate and reliable. This is driving our developments toward broader use of these elements. We're also replacing several mechanical parts made using traditional methods, such as turning and wire EDM, with additively manufactured parts. This will cut costs and speed up manufacturing and maintenance.

Manufacturing companies are consulted on the technical drawings of the revised design. Most of the manufacturing work for the new series of robots has been outsourced.

\section{Electronics}

\indent Last year, we redesigned all of our electronics from scratch to reflect the latest developments in the league and also in the industry. The main goals were:

\begin{enumerate}
    \item[$\bullet$] reliability
    \item[$\bullet$] expandability
    \item[$\bullet$] being more competitive
\end{enumerate}

After writing last year's TDP and during the design and prototyping phase, we made several changes to the architecture that can be seen in Fig. \ref{fig:electronics-architecture}. These changes will be described in the following sections.

\begin{figure}
	\centering
	\includegraphics[width=0.8\textwidth]{images/electronics-architecture.jpg}
	\caption{New main board}
	\label{fig:electronics-architecture}
\end{figure}

it's important to highlight that the new designs worked well and addressed many of the issues we faced before. However, the changes also added far more complexity to the system. Additionally, the global chip shortage, especially for the Compute Module, made it hard to get all the components we needed. As a result, we weren't able to use the new boards for half of our robots like we had planned.

\subsection{Main board}

Our current main board (Fig. \ref{fig:main-board}) was designed last year and uses a Raspberry Pi Compute Module (CM) 4 as the local compute unit on the robot. 

\begin{figure}
	\centering
	\includegraphics[width=0.8\textwidth]{images/main-board.jpg}
	\caption{New main board}
	\label{fig:main-board}
\end{figure}

We also use its 5GHz WiFi as the wireless communication link. During last year's competition we observed satisfactory results using it, with results very similar to what we had measured in the lab. Therefore we didn't follow with adding an additional nRf wireless link.

We made a design change after writing last year's TDP. Instead of using CAN, we now use Serial Peripheral Interface (SPI) to communicate with the motor controllers. The main reasons were simplicity and price. During development and testing, we couldn't find any issues with the simpler SPI bus that we hoped CAN could fix. The chip shortage caused the price of anything that speaks CAN to be much higher than in the past. The auto industry had priority over others for CAN-capable devices.

However, during the firmware implementation phase, we encountered significant challenges. We attempted to utilize the SPI bus on the Raspberry Pi for communication with the motor controllers. However, the speed of this interface fell considerably short of our expectations. We were using linux's builtin SPI driver, with Trionamic's API on top of it. This meant each byte were sent separately to the kernel causing a kernel mode switch every time. TODO: add some numbers and fact-check these.

This year we are using a manual implementation of the SPI bus, hoping that it would improve the performance.

\subsection{Motor driver}
Our motor drivers (Fig. \ref{fig:motor-driver}) are separate modules that sit on top of the main board. This greatly improves our ability to repair robots if one of the drivers fails. It also makes it easier to develop the main board and the motor drivers further separately.

\begin{figure}
	\centering
	\includegraphics[width=0.8\textwidth]{images/motor-driver.jpg}
	\caption{New motor driver module}
	\label{fig:motor-driver}
\end{figure}

We use a dedicated BLDC motor driver IC, TMC4671. It implements Field Oriented Control (FOC) for BLDC motors and includes various control methods. This offloads the local motor control functionality from the main processor to dedicated hardware, which is more reliable in terms of latency.

These drivers talk directly to the Compute Module using the SPI bus. They receive both configuration and commands, and send back sensor data including speed and position. We use the TMC-API (TODO: ref) from Trinamic to make it easier to read from and write to the TMC4671's registers.

We also use a power MOSFET driver IC, TMC6200. It drives the MOSFETs and senses the motor currents needed for the FOC algorithm. It also includes a fault detection mechanism. It shares the same SPI bus with TMC4671 and is directly connected to the Compute Module.

\subsection{Kicking board}
Our kicking board (Fig. Figure \ref{fig:mikona} uses a dedicated LT3570 flyback capacitor charger IC. It drives a BSC109N10NS3G MOSFET connected to a DA2034 transformer for this circuit.

After writing our last year's TDP, we switched to a simpler PIC microcontroller instead of the STM32. This resulted in major simplification that improved both reliablity and the cost.

We also have a high-power resistor network. It consists of three 2.4K 3W resistors. It discharges the capacitors when needed, without using the kicker magnets. The STN3N40K3 MOSFET driven by a ZXGD3009E6 is used to control the discharge.

For the actual kicking, two IGB50N60T IGBTs driven by a single IX4427MTR are used to discharge the capacitors to the kicking magnets.

It is also worth mentioning that the variable programmable resistor was removed after writing last year's TDP. This was because it was not necessary. The boost voltage rarely changes. When it does change, it usually involves a corresponding hardware change. At that time, we can simply re-calibrate the mechanical variable resistor instead.

\begin{figure}
    \centering
    \includegraphics[width=\textwidth]{images/mikona.jpg}
    \caption{The new kicking board design}
    \label{fig:mikona}
\end{figure}

\section{Software}
Our current software stack was developed in C++ in 2010 and has seen several additions and improvements over the years. This has resulted in a high performance and robust software but at the same time a difficult-to-maintain code base that is too fragile for the new changes.

This year the main focuses are to make our software:

\begin{enumerate}
    \item more robust
    \item easier to read and understand
    \item faster to iterate and extend
    \item more competitive
\end{enumerate}
In the following sections, we will describe the efforts made to reach these goals.

\subsection{Robustness}

\subsubsection{Continuous integration}
\label{section:software_ci}
This year, we created a continuous integration (CI) system based on GitHub workflows. It builds the software, performs style checks, runs automated unit tests, and optionally publishes the result as a release to GitHub. This is an essential part of our development process, as it provides several important benefits that contribute to the quality and reliability of our software.

\indent It helps ensure that the software can be built across environments without problems. This is particularly important as we currently target \textbf{Windows} (\textit{MSVC} and \textit{Clang}), \textbf{Ubuntu} (\textit{GCC} and \textit{Clang}), and \textbf{macOS} (\textit{Clang}) with a collection of library dependencies as will be described in \ref{section:software_3rdparty}. The system uses the same CMake presets that we use locally to ensure that the software is built in a consistent and reproducible way.

\indent It also performs style checks described in \ref{section:software_coding_standard} using clang-format and clang-tidy with the same configurations we use in our local integrated development environments (IDEs). This can help enforce a consistent coding style across the codebase, which can make the code easier to read and maintain.

\indent Another important aspect of the pipeline is the ability to run automated unit tests developed with \textbf{\textit{GoogleTest}} \cite{ref_3rd-party_gtest}. This can help identify problems and regressions in the software early in the development process. At the time of writing, the code coverage of these tests is not adequate and we plan to improve this over time. We also plan to introduce other ways of automated testing other than unit tests in the future, such as testing our tracker with known input and output data, and testing our defensive marking algorithm with situations from the past competitions.

\indent We also have an automatic release submission pipeline when a tag starting with \textbf{\textit{v}}, e.g. \textit{v.1.0.0} is pushed to our repositories. This automatically packages the resulting artifacts, creates a release on GitHub, and publishes the artifact along with the source code and data used to build it. Even this TDP was created using this mechanism.

\subsubsection{Third-party libraries} \label{section:software_3rdparty}
This year we started using several third-party libraries for parts of our software:
\begin{enumerate}
    \item[$\bullet$] \textbf{\textit{BehaviorTree.CPP}} \cite{ref_3rd-party_btcpp} for Behavior Trees
    \item[$\bullet$] \textbf{\textit{Asio}} \cite{ref_3rd-party_asio} for networking
    \item[$\bullet$] \textbf{\textit{Quill}} \cite{ref_3rd-party_quill} for logging
    \item[$\bullet$] \textbf{\textit{toml++}} \cite{ref_3rd-party_tomlplusplus} for configuration files
    \item[$\bullet$] \textbf{\textit{Eigen}} \cite{ref_3rd-party_eigen} for linear algebra
    \item[$\bullet$] \textbf{\textit{homog2d}} \cite{ref_3rd-party_homog2d} for 2D math
\end{enumerate}

\indent Using these libraries over our custom solutions can help improve code quality, both in terms of robustness and ease of use.

\indent These open-source projects have a proven track record and have been extensively reviewed and stabilized over time. This means they are more reliable than custom-built solutions and better suited to handle common tasks with reasonable performance and reliability.

\indent They also often come with a broader set of features that have detailed documentation, making them easier to integrate and use. This allows us to focus on implementing the core logic without worrying about the underlying infrastructure. This results in more readable code that is less prone to bugs.

\indent To make it easier to integrate other libraries, we started using a C++ dependency manager, \textbf{\textit{vcpkg}} \cite{ref_3rd-party_vcpkg}. This simplifies the installation and maintenance of third-party libraries, ensures compatibility between them, and improves reproducibility on different local machines and in the CI pipeline.

\subsubsection{Improved debugging}
One of the main weaknesses of our software in the previous competition was the lack of understanding of why the software and robots were behaving the way they were and what might be causing the problems we were seeing. We knew that by providing more detailed and informative logs, we could gain a better understanding, which could lead to faster and more effective debugging and troubleshooting.

This year, we improved our logging system with an extensible system that can output to multiple outputs, including the standard output, file, and over the network. This allows us to record the logs and later analyze any part of the runtime to better understand the behavior of the system and narrow down the problem to a specific point in time.

We have also expanded the use of our visualization GUI. Having a graphical representation of the internals of different algorithms, as well as real-time sensory data from the robots, help us better understand how different parts of the software work and make more informed decisions about how to improve the robots' performance and troubleshoot problems. The GUI also provides a more intuitive and user-friendly way to interact with the software and make configuration changes. Fig. \ref{fig_visualizer} shows an example of visualizing the internals of our path planning.

\begin{figure}
    \includegraphics[width=\textwidth]{images/visual1.png}
    \caption{A demonstration of the ERRT path plan in the visualization GUI.}
    \label{fig_visualizer}
\end{figure}

The GUI is implemented in Python and receives the visualization data over the network. This means that it can be run on any computer in the same network and receive the data from any node in the system, including the tracker, the soccer ball, and even the robots' embedded firmware.


\subsection{Readability}

\subsubsection{Architecture}
Our current software is a single monolithic application that handles world state estimation, AI, and robot motion planning. This has the advantage of allowing us to easily change the flow of data between different parts. But it forces us to implement everything in C++ to produce a single application that runs on a single machine. Another side effect of such a monolithic design was that it encouraged more coupling between the soccer and vision parts of the software, making it harder to make changes to either.

\begin{figure}
	\centering
	\includegraphics[width=0.9\textwidth]{images/software-architecture.jpg}
	\caption{The new software architecture}
	\label{fig:software-architecture}
\end{figure}

The goal this year is to refactor the code base into separate parts that are connected via the network as shown in Fig. \ref{fig:software-architecture}. This will allow us to move the lower-level motion planning and skill execution to the robot's local processor and develop the graphical user interface (GUI) using other technologies.

At the time of writing, these efforts are still ongoing, but we are confident that we will be able to transition to the new stack in time for RoboCup 2023.

\subsubsection{Coding standards}
\label{section:software_coding_standard}
One of the most important observations in the past has been the complexity of the C++ language and the large number of problems that can arise when it is used incorrectly. This is especially important to us because the code is typically developed by students who do not have extensive experience with the language.

To address this issue, we have chosen a coding standard \cite{ref_cppbestpractices} and agreed on a set of style and naming conventions to be used throughout the code base. These will improve readability and maintainability, and promote consistency. These style checks are implemented as \textbf{\textit{clang-tidy}} and \textbf{\textit{clang-format}} configurations which can be used both locally within the IDEs, and as an automated gate in our CI pipeline as described in \ref{section:software_ci} when creating a pull-request to the main branch.

As a first step in implementing the coding standard, we have begun to move the code base to a more modern C++ revision (C++20 at the time of this writing). It includes as an example moving away from the use of raw pointers in favor of smart pointers. This can help simplify memory management, improve code clarity and maintainability, and reduce the risk of errors related to object ownership and lifetime.

We are also in the process of making the code base warning-free and treating the new warnings as bugs. These warnings are issued by the compiler when it detects potential problems in the code that could lead to bugs or undefined behavior based on the C++ Core Guidelines \cite{ref_cppcoreguidelines} or other references. Treating them as errors force developers to address them and fix potential problems early in the development process, which can help prevent the accumulation of technical debt and reduce the risk of introducing bugs later.

\subsection{Extensibility}

\subsubsection{Issues with Finite-State Machines}
Our current AI is based on finite-state machines (FSMs) \cite{ref_ETDP2020}. While this structure made it easier to decompose robot behavior into distinct states and implement each event as a transition between two states, it made it difficult to reuse states and increased the number of transitions exponentially. It also became very difficult to debug or extend the FSM to handle new behaviors and implement fallback tactics.

When implementing each state of an FSM, the developer must consider the other states and how they will transition to the current state and how it will transition to the next state. For a state machine with only a few states, this is easier to handle. However, considering the number of robots and the different tactics that the opponent can use, the developer is required to implement a large number of states to handle every possible situation in the FSM. This can be a significant challenge for the developer to implement all the states and their relationships.

\subsubsection{Behavior trees}
To overcome these problems, we switched to Behavior Trees (BT) this year. BTs are hierarchical structures composed of nodes representing actions, conditions, or other behaviors, and their connections define the order in which those behaviors should be executed. BT can be used to develop AI for soccer-playing robots by providing a framework for creating complex decision-making algorithms that can handle multiple goals and constraints.

\begin{table}
    \centering
    \footnotesize
    \caption{Low-level behaviors.}
    \label{tab:low-level-behaviors}
    \begin{tabular}{|p{0.5\textwidth}|p{0.45\textwidth}|}
        \hline
        \bf{Behavior} & \bf{Description} \\
        \hline
        {\itshape Navigate(destination, orientation, profile)} & Navigate the robot to \textit{destination} with the provided velocity \textit{profile} while avoiding obstacles.\\
        {\itshape Chip(power)} & Request a chip kick with the passed \textit{power} to be executed when the ball is detected.\\
        {\itshape Direct(power)} & Request a direct kick with the passed \textit{power} to be executed when the ball is detected.\\
        \hline
        {\itshape Obstacles(situation)} & Returns obstacles for the robot for a given \textit{situation} in the game.\\
        {\itshape Face(point)} & Returns an orientation so that the robot faces the passed \textit{point}.\\
        {\itshape Mark(opponent)} & Returns the defensive marking position for the given \textit{opponent} that is either between our goal and the \textit{opponent}, or the ball and the \textit{opponent}.\\
        {\itshape FetchBall(point)} & Returns the position on line which the ball is moving on and most close to the \textit{point}, and an orientation to fetch the ball.\\
        {\itshape OneTouch(point, target)} & Returns the position on line which the ball is moving on and most close to the \textit{point}, and an orientation to kick the ball towards the \textit{target}.\\
        \hline
    \end{tabular}
\end{table}

Table \ref{tab:low-level-behaviors} shows the low-level behaviors that can be used as building blocks to create more complex behaviors. They are self-contained and easy to understand, so they can be developed and tested once, which is much less time-consuming than maintaining complex and interconnected systems developed together.

Fig. \ref{fig:bt-move} shows an example BT that is built upon the lower-level behaviors described earlier in table \ref{tab:low-level-behaviors}. Note that this tree is stored as an \textit{XML} file that the software can load and reload at runtime, without a need to recompile. This greatly improves the iteration times.

\begin{figure}
	\centering
	\includegraphics[width=0.95\textwidth]{images/ball_placement_bt.png}
	\caption{A sample Behavior Tree for ball placement.}
	\label{fig:bt-move}
\end{figure}

By using BTs in the development of our AI, we can create flexible, modular, and easy-to-maintain decision-making algorithms that can handle the complexity and variability of the games. The hierarchical structure of the BT also allows for a high degree of customization and adaptability, making it easier for us to adjust our strategy based on the specific conditions of each game.

\subsubsection{Config system}
Previously, all parameters were hard-coded into the C++ code. This made it time-consuming to change anything, resulting in unacceptably long iteration times. Using a config file instead makes it easier to maintain and update the values, as they can be changed in a central location instead of having to search for each instance of the hard-coded value. It also makes the code more flexible and reusable, since you can easily switch between different configurations without having to change the code itself. This also makes it easier to test different scenarios without restarting the system.

This year we started using \textbf{\textit{toml}} \cite{ref_toml} as our configuration format, along with a \textbf{\textit{json schema}} \cite{ref_json-schema} that we generate based on our C++ config structure to validate the toml file in both text editors and as a CI step. This schema is also used by our GUI to create a persistent config editor with the correct types, names, and defaults, even in the absence of a config file. The system uses a layered architecture; multiple toml configs, both from disk files and from the network, are fed into the system, each with a priority field. The configuration system then returns a single compiled configuration that all nodes in the system can use (Fig. \ref{fig:software-architecture}). The system can also prompt the nodes in the system to persist the received config.

\subsubsection{Improving build time}
One of the common pitfalls of any C++ code base is the time it takes to build the project. This problem became more apparent as our software grew in size and dependencies. Before doing any optimizations, it took about 15 minutes to do a complete rebuild on a modern desktop CPU with 6 physical and 12 logical processors.

As a first step, we enabled multi-threaded compilation with \textbf{\textit{/MP}} flag on MSVC and using \textbf{\textit{Ninja}} \cite{ref_ninja} instead of make. This greatly reduced the build time to about 3 minutes. But it was still unacceptable for the iteration times we were looking for.

\begin{figure}
	\centering
	\includegraphics[width=\textwidth]{images/time-trace.png}
	\caption{The output of the \textit{-ftime-trace} for one of the source files.}
	\label{fig:time-trace}
\end{figure}

We then used the \textit{-ftime-trace} flag with Clang to see where the time is spent during the build. As shown in Fig. \ref{fig:time-trace}, most of the time is spent in the frontend, in this example about \%96 of it. Upon closer inspection, it is clear that this is caused by excessive use of include files, which the compiler frontend has to process multiple times.

To solve this problem, we started using \textit{unity build}, so that the headers are processed once for a set of source files. This resulted in a massive reduction in build times. A full rebuild took about 20 seconds on the same machine. But touching a single source file still triggered the compilation of a set of files, which is inefficient.

The last step to improve build time was to use \textit{pre-compiled header} (PCH). All header files from the 3rd party libraries, and the common code used by both vision and soccer modules except the generated protobuf header files are pre-compiled. This is then used by the compiler when compiling the vision and soccer modules. This way, the time to do a full rebuild is almost the same as the unity build. But the time to build the modules when only a few source files are changed became about 2 seconds, which shows a reduction of about \%80 compared to the unity build.

We currently use PCH locally, and unity builds together with PCH for our CI pipeline.

\subsection{Competitiveness}
As mentioned earlier, the main focus this year has been to improve the foundation of our code base to make it easier to build more competitive software on top of it. We expect to see more soccer-related development in the coming years. The following sections describe the work we expect to be done in time for RoboCup 2023.

\subsubsection{Local processing on robots}
Currently, we run all the behaviors in the main software on a computer, and then send the target commands in the form of global target velocity, orientation, and kick commands to the robots. The robot's local processor then performs sensor fusion on the orientation processed by the vision and the local Inertial measurement unit (IMU). This greatly improves the quality of the predicted orientation.

To extend this idea, we plan to do more sensor fusion on the robot. We will use the output of the four rotary encoders connected to the wheels, the output of the IMU, the proximity sensor array around the robot, and the filtered world state received from the tracker.

In addition, we want to move most of the low-level behaviors shown in table \ref{tab:low-level-behaviors} to the robot's local processor. Combined with local sensor fusion, this results in improved navigation efficiency and safety, even in the presence of excessive vision noise and wireless link latency variations.

\newpage
\begin{thebibliography}{8}
\bibitem{ref_ETDP2020}
Immortals 2020 Extended Team Description Paper, \url{https://ssl.robocup.org/wp-content/uploads/2020/03/2020\_ETDP\_Immortals.pdf}.

\bibitem{ref_ETDP2019}
Immortals 2019 Team Description Paper, \url{https://ssl.robocup.org/wp-content/uploads/2019/03/2019\_ETDP\_Immortals.pdf}.

\bibitem{ref_ETDP2018}
O. Najafi Koopai, M.A. Ghasemieh, M. Khanloghi. Immortals 2018 Team Description Paper.

\bibitem{ref_tigers_etdp_2020}
Ryll, Andre and Jut, Sabolc. TIGERS Extended Team Description for RoboCup 2020


\bibitem{ref_github}
Immortals Open Source Project. \url{https://github.com/Immortals-Robotics}.

\bibitem{ref_cppbestpractices}
Turner, Jason. "C++ Best Practices"

\bibitem{ref_cppcoreguidelines}
Stroustrup, Bjarne and Sutter, Herb. "C++ Core Guidelines"

\bibitem{ref_toml}
TOML, A config file format for humans. \url{https://toml.io/}.

\bibitem{ref_json-schema}
JSON Schema, a declarative language that allows you to annotate and validate JSON documents. \url{https://json-schema.org/}.

\bibitem{ref_ninja}
Ninja, a small build system with a focus on speed. \url{https://ninja-build.org/}.

% 3rd-party libraries
\bibitem{ref_3rd-party_gtest}
GoogleTest, Google Testing and Mocking Framework \url{https://github.com/google/googletest}

\bibitem{ref_3rd-party_vcpkg}
vcpkg, cross-platform C/C++ dependency manager from Microsoft \url{https://vcpkg.io/}

\bibitem{ref_3rd-party_btcpp}
BehaviorTree.CPP, a C++ library to build Behavior Trees. \url{https://www.behaviortree.dev/}

\bibitem{ref_3rd-party_asio}
Asio, cross-platform C++ library for network and low-level I/O programming \url{https://think-async.com/Asio/}

\bibitem{ref_3rd-party_quill}
Quill, asynchronous Low Latency C++ Logging Library \url{https://github.com/odygrd/quill}

\bibitem{ref_3rd-party_tomlplusplus}
toml++, header-only TOML config file parser and serializer for C++17. \url{https://github.com/marzer/tomlplusplus}

\bibitem{ref_3rd-party_eigen}
Eigen, C++ template library for linear algebra \url{https://eigen.tuxfamily.org/}

\bibitem{ref_3rd-party_homog2d}
homog2d, C++ 2D geometry library \url{https://github.com/skramm/homog2d}

\end{thebibliography}
\end{document}
