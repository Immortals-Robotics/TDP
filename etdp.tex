% This is samplepaper.tex, a sample chapter demonstrating the
% LLNCS macro package for Springer Computer Science proceedings;
% Version 2.20 of 2017/10/04
%
\documentclass[runningheads]{llncs}
%
\usepackage{subcaption}
\usepackage{graphicx}
\usepackage{placeins}
\usepackage{float}
% Used for displaying a sample figure. If possible, figure files should
% be included in EPS format.
%
% If you use the hyperref package, please uncomment the following line
% to display URLs in blue roman font according to Springer's eBook style:
% \renewcommand\UrlFont{\color{blue}\rmfamily}

\begin{document}
%
\title{Immortals 2023 Extended Team Description Paper}
\titlerunning{Immortals 2023 ETDP}

\author{Ali Salehi \and
Mohammad Tabasi \and
Omid Najafi \and
MohammadHossein Fazeli \and
MohammadAli Ghasemieh \and
MohammadReza Niknezhad \and
Mustafa Talaeezadeh \and
Ali Amoozandeh Nobaveh}
%
\authorrunning{Immortals Robotics}
%
\institute{
\url{http://www.immortals-robotics.com}
}
%
\maketitle              % typeset the header of the contribution
%
\begin{abstract}
%The abstract should briefly summarize the contents of the paper in
%15--250 words.
This paper describes the recent work that has been done by the Immortals Robotics Team for the upcoming RoboCup 2023 competition in Bordeaux, France.

\keywords{RoboCup 2023 \and Small Size League}
\end{abstract}
%
%
%
\section{Introduction}
The Immortals Small Size League team was founded in 2008 and participated for the first time in RoboCup 2009 in Graz. The team consists of computer and electrical engineers and currently focuses on the Small Size League.

There have been some changes in the mechanics and electronics of our robots (Fig. \ref{fig:std_robot}) in the last few years. The process can be seen in the previous TDPs and ETDPs~\cite{ref_ETDP2019}.
 
This year, the team focused on modernizing the electronics and software architecture. Efforts were also made to resolve issues observed during recent competitions, including RoboCup 2018 in Montréal. In addition to the current robot, there is a 3D-printed prototype robot that was presented in 2018 and has been improved and tested since then. The goal is to achieve a modular, flexible, and reliable platform that would reduce the maintenance and future development costs of the robots.

\begin{figure}
    \centering
    \begin{subfigure}[b]{0.45\textwidth}
         \centering
         \includegraphics[width=\textwidth]{images/std_robot.jpeg}
         \caption{Standard}
         \label{fig:robot_std}
    \end{subfigure}
    \hfill
    \begin{subfigure}[b]{0.5\textwidth}
        \centering
        \includegraphics[width=\textwidth]{images/printed_robot.jpeg}
        \caption{3D-printed}
        \label{fig:robot_printed}
    \end{subfigure}
    \caption{Immortals current robots.}
    \label{fig:std_robot}
\end{figure}

\section {Mechanics}
Due to the optimal design of the previous robots, the mechanical design team decided not to overhaul the entire system, but rather to modify design details to improve the manufacturability and durability of the parts. In addition, recent advances in additive manufacturing techniques, which have resulted in more accurate and reliable 3D-printed parts, are driving our developments toward broader use of these elements and the replacement of several mechanical parts currently manufactured using conventional methods, such as turning and wire EDM, with additively manufactured parts to reduce costs and increase manufacturing and maintenance speeds.

The manufacturing companies are consulted on the technical drawings of the revised design, and most of the manufacturing work for the new series of robots has been outsourced.

%%%%%%%%%%%%

\section{Electronics}

In 2018, changes were made to the electronics to modernize the designs and replace the old parts with their new counterparts. We tested them during RoboCup 2018, and the results show a solid improvement in reliability while reducing production costs. Currently, all robots use these circuits. The reader is referred to this team's previous year's TDP [1] for more details on the main circuit.\\
\indent This year, we're planning to redesign all of our electronics from scratch to reflect the latest developments in the league and also in the industry. The main goals are:

\begin{enumerate}
    \item[$\bullet$] reliability
    \item[$\bullet$] expandability
    \item[$\bullet$] being more competitive
\end{enumerate}

\begin{figure}
	\centering
	\includegraphics[width=0.8\textwidth]{images/electronics-architecture.jpg}
	\caption{The new electronics architecture}
	\label{fig:electronics-architecture}
\end{figure}

The new architecture can be seen in Fig. \ref{fig:electronics-architecture}. It should be mentioned that at the time of this writing, the design work is still ongoing and we don't have the final boards manufactured. We hope to equip at least half of our robots with the new parts in RoboCup 2023. We will publish the designs on our GitHub page~\cite{ref_github} shortly after the competition.

\subsection{Main board}
The current main board used by the team was designed in 2010 and has been used in its original form ever since, except for minor changes. It uses a Xilinx Spartan3 FPGA as the main and only processor. A soft processor, TASKING TSK3000, is used inside the FPGA to handle more sequential logic, while the FPGA itself is used to read encoders, drive BLDC motors, drive the boost converter, etc.
While this design is flexible and comparatively cheap to produce, it has shown its age in recent years. The main drawbacks are:
\begin{enumerate}
    \item[$\bullet$] The TSK3000 runs at about 36 MHz and is far too limited in its current configuration to develop more sophisticated local processing and motion planning. Some effort has been made in the past to move parts of the performance-critical C code to the logic gates, but this would make the implementation more difficult to modify and extend. On the other hand, the debugging workflow was too restrictive, and any changes to the code required a complete rebuild of the FPGA project. All of these factors resulted in the team using pretty much the same framework for several years without being able to make major changes.
    \item[$\bullet$] BLDC motor commutation is a simple 6-step trapezoidal commutation. It is easy to implement, but is inefficient and causes high torque ripples. Implementing a more sophisticated method, such as Field Oriented Control, requires massive changes to the PCB.
    \item[$\bullet$] We use the nRF24L01 chip for wireless communications, with a custom payload layout on top of its Enhanced ShockBurst (ESB) protocol. This gives us a lot of flexibility, but because it is a low-level protocol, we have to add any higher-level functionality we need, such as discovery.
    \item[$\bullet$] The waveform needed to drive the boost converter in the kicking board is generated by the FPGA. This meant that we were free to change it to suit our needs, but in practice, it was considered too fragile.
    \item[$\bullet$] There are no current protections on the board. Any malfunction in the board itself or in any other part of the robot, such as a stuck wheel, will cause damage to the parts. This greatly reduces reliability, increases maintenance costs, and causes damage to the battery.
\end{enumerate}

To resolve these issues the work started on designing a new main board from scratch. The main features are:
\begin{enumerate}
    \item[$\bullet$] Raspberry Pi Compute Module (CM) 4 as the local compute unit on the robot. We intend to move parts of the skill execution, data fusion and prediction, and motion planning to it.
    \item[$\bullet$] Compute Module's 5GHz WiFi as the wireless communication link. This will greatly increase the bandwidth and capabilities of the link and allow us to add robots as regular links to our software stack. This will allow them to receive world state and the AI output necessary to perform local skills.\\
    The latency characteristics of using WiFi instead of a low-level protocol in the lab environment were satisfactory. Using the standard PCB antenna at a distance of 20m from the access point, we were able to achieve a latency of ~2ms with a data loss of ~\%3. The latency requirements will be more relaxed after moving more processing to the robot's local processor. However, we are still considering adding a separate nRF chip for latency-sensitive processing if the new approach causes problems.
    \item[$\bullet$] CAN protocol to connect the main board to external boards, including the kicking board, motor drivers, proximity sensors, and battery monitoring. This will give us a more robust and flexible base to build on.
\end{enumerate}

\subsection{Motor driver}
In past competitions, the motor control circuitry was one of the most common points of failure for the robot. Since they were on the same PCB, repairing them would require a complete reflow of the broken parts. In more severe cases, such as when the traces are damaged, especially on the inner layers, this could mean that the board becomes irreparable.\\
This year, we decided to design separate modular motor driver boards that will be placed on the main board for each motor. This will greatly improve our ability to repair robots if one of the drivers fails. It would also make it easier to develop the main board and the motor drivers separately.\\
This new driving board is based on:
\begin{enumerate}
    \item[$\bullet$] A dedicated brushless DC (BLDC) motor driver IC, TMC4671. It implements Field Oriented Control (FOC) for BLDC motors and includes various control methods. This offloads the local motor control functionality from the main processor to dedicated hardware, which is more reliable in terms of latency. It has an SPI interface to receive both configuration and commands and to send back sensor data including speed and position.
    \item[$\bullet$] A power MOSFET driver IC, TMC6200. It drives the MOSFETs and senses the motor currents needed for the FOC algorithm. It also includes a fault detection mechanism.
    \item[$\bullet$] A small Arm processor, STM32F042G6Ux, acts as a CAN client link to the main processor. It communicates with both the TMC4671 and the TMC6200 via SPI and also reads the current sensor.
\end{enumerate}

\subsection{Kicking board}
In previous years, we used a boost converter driven by the FPGA from the main board. There were also two discharge pins and one charge pin connected directly to the FPGA's IO pins. These were major problems with this design both in terms of reliability and charge performance.\\

\begin{figure}
	\centering
    \begin{subfigure}[b]{0.45\textwidth}
         \centering
         \includegraphics[width=\textwidth]{images/mikona_top.png}
         \caption{top-view}
         \label{fig:mikona_top}
    \end{subfigure}
    \hfill
    \begin{subfigure}[b]{0.45\textwidth}
        \centering
        \includegraphics[width=\textwidth]{images/mikona_bottom.png}
        \caption{bottom-view}
        \label{fig:mikona_bottom}
    \end{subfigure}

    \caption{The new kicking board design}
    \label{fig:mikona}
\end{figure}

This year we redesigned the kicking board (Fig. \ref{fig:mikona}) with the following features:
\begin{enumerate}
    \item[$\bullet$] A dedicated LT3570 flyback capacitor charger IC is used. This simplifies the design while improving performance and reliability. We use the DA2034 transformer and the BSC109N10NS3G MOSFET for this circuit.
    \item[$\bullet$] A STM32F042G6Ux MCU is used on the board to handle the CAN protocol to the main board and to control the charger IC, variable resistors, and discharge IGBTs.
    \item[$\bullet$] A high-power resistor network consisting of three 2.4K 3W resistors is added to the board to discharge the capacitors when needed without using the kicker magnets. The STN3N40K3 MOSFET driven by a ZXGD3009E6 is used to control the discharge.
    \item[$\bullet$] Two IGB50N60T IGBTs driven by a single IX4427MTR are used to discharge the capacitors to the kicking magnets.
    \item[$\bullet$] An MCP4562 variable resistor to set the target voltage of the flyback converter. This allows us to change the voltage by simply changing a configuration variable in the main software and communicating it down to the kicking board.
\end{enumerate}

\section{Software}
Our current software stack was developed in C++ in 2009 and has seen several additions and improvements over the years. This has resulted in a high performance and robust software but at the same time a difficult-to-maintain code base that is too fragile for the new changes.

This year the main focuses are to make our software:

\begin{enumerate}
    \item more robust
    \item easier to read and understand
    \item faster to iterate and extend
    \item more competitive
\end{enumerate}
In the following sections, we will describe the efforts made to reach these goals.

\subsection{Improving robustness}

\subsubsection{Continuous integration}
This year, we created a continuous integration (CI) system based on GitHub workflows. It builds the software, performs style checks, and runs automated unit tests, and optionally publishes the result as a release to GitHub. This is an essential part of our development process, as it provides several important benefits that contribute to the quality and reliability of our software.

\indent It helps ensure that the software can be built across environments without problems. This is particularly important as we currently target \textbf{Windows} (\textit{MSVC} and \textit{Clang}), \textbf{Ubuntu} (\textit{GCC} and \textit{Clang}), and \textbf{macOS} (\textit{Clang}) with a specific configuration of library dependencies. The system uses the same CMake presets that we use locally to ensure that the software is built in a consistent and reproducible way.

\indent It also performs style checks using clang-format and clang-tidy with the same configurations we use in our local IDEs. This can help enforce a consistent coding style across the codebase, which can make the code easier to read and maintain, and can also help prevent problems that can arise from inconsistent code formatting, such as merge conflicts.

\indent Another important aspect of the pipeline is the ability to run automated unit tests developed with \textbf{\textit{GoogleTest}}~\cite{ref_3rd-party_gtest}. This can help identify problems and regressions in the software early in the development process. This can help prevent problems from being introduced into the software, and can help ensure that the software meets the project's requirements and specifications. At the time of writing, the code coverage of these tests is not adequate and we plan to improve this over time. We also plan to introduce more automated tests other than unit tests in the future, such as testing our tracker with known input and output data.

\indent We also have an automatic release submission pipeline when a tag starting with \textbf{\textit{v}} is pushed to our repositories. This automatically packages the resulting artifacts, creates a release on GitHub, and publishes the artifact along with the source code and data used to build it. Even this TDP was created using this mechanism.

\subsubsection{Third-party libraries}
This year we started using several third-party libraries for parts of our software:
\begin{enumerate}
    \item[$\bullet$] \textbf{\textit{Asio}}~\cite{ref_3rd-party_asio} for networking 
    \item[$\bullet$] \textbf{\textit{Quill}}~\cite{ref_3rd-party_quill} for logging
    \item[$\bullet$] \textbf{\textit{toml++}}~\cite{ref_3rd-party_tomlplusplus} for configuration files
    \item[$\bullet$] \textbf{\textit{Eigen}}~\cite{ref_3rd-party_eigen} for linear algebra
    \item[$\bullet$] \textbf{\textit{homog2d}}~\cite{ref_3rd-party_homog2d} for 2D math
\end{enumerate}

\indent Using these libraries over our custom solutions can help improve code quality, both in terms of robustness and ease of use.

\indent These open-source projects have a proven track record and have been extensively reviewed and stabilized by experts over time. This means they are more reliable than custom-built solutions and better suited to handle common tasks with reasonable performance and reliability.

\indent They also often come with a broader set of features that have detailed documentation, making them easier to integrate and use. This allows developers to focus on implementing the core logic without worrying about the underlying infrastructure. This results in more readable code that is less prone to bugs.

\indent To make it even easier to use other libraries, we started using a C++ dependency manager, \textbf{\textit{vcpkg}}~\cite{ref_3rd-party_vcpkg}. This simplifies the installation and maintenance of third-party libraries, ensures compatibility between them, and improves reproducibility on different local machines and in the CI pipeline.

\subsubsection{Improved debugging}
One of the main weaknesses of our software in the previous competition was the lack of understanding of why the software and robots were behaving the way they were and what might be causing the problems. We knew that by providing more detailed and informative logs, we could gain a better understanding, which could lead to faster and more effective debugging and troubleshooting.

This year, we improved our logging system with an extensible system that can output to the command line, to a file, and over the network. This allows us to later analyze any part of the runtime to better understand the behavior of the system and narrow down the problem to a specific point in time.

We have also expanded the use of our visualization GUI. Having a graphical representation of the internals of different algorithms, as well as real-time sensory data from the robots, helps us better understand how different parts of the software work and make more informed decisions about how to improve the robots' performance and troubleshoot problems. The GUI also provides a more intuitive and user-friendly way to interact with the software and make configuration changes. Fig~\ref{fig_visualizer} shows an example of visualizing the internals of our ERRT path planning.

\begin{figure}
    \includegraphics[width=\textwidth]{images/visual1.png}
    \caption{A demonstration of the ERRT path plan in the visualization GUI.}
    \label{fig_visualizer}
\end{figure}

The GUI is implemented in Python and receives the visualization data over the network. This means that it can be run on any computer in the same network and receive the data from any node in the system, including the tracker, the soccer ball, and even the robots' embedded firmware.


\subsection{Improving readability}

\subsubsection{Architecture}
Our current software is a single monolithic application that handles world state estimation, AI, and robot motion planning. This has the advantage of allowing us to easily change the flow of data between different parts. But it forces us to implement everything in C++ to produce a single application that runs on a single machine. Another side effect of such a monolithic design was that it encouraged more coupling between the soccer and vision parts of the software, making it harder to make changes to either.

\begin{figure}
	\centering
	\includegraphics[width=0.9\textwidth]{images/software-architecture.jpg}
	\caption{The new software architecture}
	\label{fig:software-architecture}
\end{figure}

The goal this year is to refactor the code base into separate parts that are connected via the network as shown in Fig. \ref{fig:software-architecture}. This will allow us to move the lower-level motion planning and skill execution to the robot's local processor and develop the graphical user interface (GUI) using other technologies.

At the time of writing, these efforts are still ongoing, but we are confident that we will be able to transition to the new stack in time for RoboCup 2023.

\subsubsection{Coding standards}
One of the most important observations in the past has been the complexity of the C++ language when used incorrectly. This is especially important to us because the code is typically developed by students who do not have extensive experience with the language.

To address this issue, we have chosen a coding standard~\cite{ref_cppbestpractices} and agreed on a set of style and naming conventions to be used throughout the code base. These will improve readability and maintainability, and promote consistency. These style checks are implemented as \textbf{\textit{clang-tidy}} and \textbf{\textit{clang-format}} configurations which can be used both locally within the IDEs, and as an automated gate in our CI pipeline when creating a pull-request to the main branch.

As a first step in implementing the coding standard, we have begun to move the code base to a more modern C++ revision (C++20 at the time of this writing). This means moving away from the use of raw pointers in favor of smart pointers. This can help simplify memory management, improve code clarity and maintainability, and reduce the risk of errors related to object ownership and lifetime.

We are also in the process of making the code base warning-free and treating the new warnings as bugs. These warnings are issued by the compiler when it detects potential problems in the code that could lead to bugs or undefined behavior. Treating them as errors force developers to address them and fix potential problems early in the development process, which can help prevent the accumulation of technical debt and reduce the risk of introducing bugs later. Some of these warnings are based on the C++ Core Guidelines~\cite{ref_cppcoreguidelines}, which should be used in most C++ code to avoid common pitfalls seen in other projects.

% Written by Omid (We need to merge this with the above soon.

% Our team's main focus this year has been on designing a more effective AI system for our robot, with a particular emphasis on improving our debugging process. To achieve this goal, we have redesigned our AI system from the ground up, using C++20 and a modular architecture consisting of three classes: ExternalWorld, WorldState, and Player.

% The ExternalWorld class has different subclasses, each of which works in one of the following ways:

% \begin{itemize}
% \item[$\bullet$] \textbf{RealWorld:} This subclass receives data from a defined SSL-vision and SSL-refbox, and sends robot commands to a defined sender (which forwards the commands to the desired robot).
% \item[$\bullet$] \textbf{SimulatedWorld:} This subclass receives data from a defined SSL-vision and SSL-refbox simulator, or any virtual vision or refbox software that adheres to the official protocols of the SSL Communication Protocol, and sends it to a simulator with its own defined protocol.
% \item[$\bullet$] \textbf{LogfileWorld:} This subclass receives data from a given logfile of a game. In this mode, the robots act according to the information contained in the logfile without any control. This mode is useful for observing how the player acts in various scenarios. For example, the debugger can show what path each robot would take if it were being controlled by our AI or what strategies to use when the gameplay enters a specific state defined in the logfile.
% \end{itemize}

% The WorldState class contains all the data describing the state of the game. This includes information about the locations of the robots and ball, as well as the current score and other game details. The ExternalWorld class feeds data to the WorldState module, which uses it to update its internal state.

% The Player class is responsible for navigating the robot and generating commands for each robot. This is where the actual AI algorithms are implemented, including path planning, obstacle avoidance, and ball tracking. The Player class uses the data from the WorldState module to make decisions about how to move the robot and interact with other objects in the game.

% We have also added both text-based and graphical loggers to our AI system to improve our debugging process. The text-based logger writes output directly to stdout, allowing us to more easily probe exceptions and debug our code. The graphical logger provides us with a visual representation of what the AI software "sees" and how it makes decisions based on the information it has. By utilizing these logging tools, we hope to more effectively debug and refine our AI system.

% Throughout the development process, we focused on making our AI software as open and flexible as possible, to allow for easy implementation of new and innovative ideas. We also tested and debugged our software extensively, to ensure that it performs reliably and effectively in different game situations.

%%%%%%%%%%%

\subsection{Improving extensibility}
Our current AI is based on finite-state machines (FSMs)~\cite{ref_ETDP2020}. While this structure made it easier to decompose robot behavior into distinct states, it made it difficult to reuse states and increased the number of transitions exponentially. To overcome these problems, we switched to Behavior Trees (BT) this year.

\subsubsection{Behavior trees}
Behavior Trees (BT) are hierarchical structures composed of nodes representing actions, conditions, or other behaviors, and their connections define the order in which those behaviors should be executed. BT can be used to develop AI for soccer-playing robots by providing a framework for creating complex decision-making algorithms that can handle multiple goals and constraints.

\begin{table}[H]
    \footnotesize
    \caption{Example of low-level behaviors.}
    \label{tab:low-level-behaviors}
    \begin{tabular}{|p{7.5cm}|p{6cm}|}
        \hline
        \bf{Behavior} &  \bf{Explanation} \\
        \hline
        {\itshape Navigate2Point(robot, destination, maxSpeed)} & Navigate the robot to a destination.\\
        {\itshape ERRTNavigate2Point(robot, destination, maxSpeed)} & Navigate the robot while avoiding obstacles.\\
        {\itshape Mark(robot, oppRobot)} & Position between the goal and an opponent robot.\\
        {\itshape FetchBall(robot, point)} & Navigate the robot to a position on line which the ball is moving on and most close to the \textit{point}.\\
        {\itshape OneTouchDirect(robot, point, target)} & Navigate the robot to a position on line which the ball is moving on and most close to the \textit{point}. Kick the ball towards the \textit{target}.\\
        {\itshape CircleBall(robot, radius, angle)} & Position robot on a circle around the ball in a specific angle.\\
        {\itshape Face(robot, point)} & Face the robot towards a point.\\
        {\itshape Chip(robot, power)} & Robot should perform a chip kick whenever the ball was intercepted.\\
        {\itshape Direct(robot, power)} & Robot should perform a direct kick whenever the ball was intercepted.\\
        {\itshape CircleKickBall(robot, target, power)} & Position robot on a circle around the ball and towards the \textit{target}, then kick the ball.\\
        \hline
    \end{tabular}
\end{table}

Table~\ref{tab:low-level-behaviors} shows the low-level behaviors that can be used as building blocks to create more complex behaviors as shown in Fig. \ref{fig:bt-move}. They are self-contained and easy to understand, so they can be developed and tested once, which is much less time-consuming than maintaining complex and interconnected systems developed together.

\begin{figure}
	\centering
	\includegraphics[width=0.8\textwidth]{images/bt-move.png}
	\caption{A sample Behavior Tree for moving to a point}
	\label{fig:bt-move}
\end{figure}

By using BTs in the development of our AI, we can create flexible, modular, and easy-to-maintain decision-making algorithms that can handle the complexity and variability of the games. The hierarchical structure of the BT also allows for a high degree of customization and adaptability, making it easier for us to adjust our strategy based on the specific conditions of each game.

\subsubsection{Config system}
Previously, all parameters were hard-coded into the C++ code. This made it time-consuming to change anything, resulting in unacceptably long iteration times. Using a config file instead makes it easier to maintain and update the values, as they can be changed in a central location instead of having to search for each instance of the hard-coded value. It also makes the code more flexible and reusable, since you can easily switch between different configurations without having to change the code itself. This also makes it easier to test different scenarios without restarting the system.

This year we started using \textbf{\textit{toml}}~\cite{ref_toml} as our configuration format, along with a \textbf{\textit{json schema}}~\cite{ref_json-schema} that we generate based on our C++ config structure to validate the toml file in both text editors and as a CI step. This schema is also used by our GUI to create a persistent config editor with the correct types, names, and defaults, even in the absence of a config file. The system uses a layered architecture; multiple toml configs, both from disk files and from the network, are fed into the system, each with a priority field. The configuration system then returns a single compiled configuration that all nodes in the system can use (Fig. \ref{fig:software-architecture}) . The system can also prompt the nodes in the system to persist the received config.

% \subsection{Analyzing}
% With the tools and designs which were introduced above, it is now possible to demonstrate different features of this project. Below, we will describe the analyzing process in the AI with an example. In the example the goal is to find the best spot in order for a robot to perform a one-touch kick\footnote{A one-touch kick is a robots action where a ball is kicked immediately it touches the front of the robot.}.

% There are many parameters to notice while performing a successful one-touch kick (e.g. Initial Ball Velocity, Robots Angle, Velocity of the robot, Target position of the kick). A simple solution to find the optimum values for the parameters is to run tests with different initializations of the parameters. Here, the tests are performed in grSim~\cite{ref_grsim}.

% For this example at first, a ball and two robots are stationed at defined locations and a target position is randomly picked in a defined window.

% Second, after waiting for a while, one of the robots moves toward the ball and kicks it towards the target position. Meanwhile, the other robot tries to reach to the target position.

% Third, After the ball has been moved the second robot is commanded to perform a one touch kick and direct the ball to the center of the goal. This process continues until the ball passes the fields side line or the ball stops moving. If the ball enters the goal, the target position is tagged as a success, in other cases it is tagged as a fail. After this the \textit{round} counter increases by one and the process is repeated from the first state.

% At last, After 500 rounds the process is finished and the results are shown in the debugging tools (i.e. the graphical visualizer).

% Fig~\ref{fig_ANALYZE_SING_ITR} shows how a single round is performed.

% \begin{figure}
% \centering
% \includegraphics[height=5cm]{images/Analyze_State1.png}
% \includegraphics[height=5cm]{images/Analyze_State2.png}
% \includegraphics[height=5cm]{images/Analyze_State3.png}
% \includegraphics[height=5cm]{images/Analyze_State4.png}\caption{The analysis process shown in the graphical visualizer. Starting from left.} \label{fig_ANALYZE_SING_ITR}
% \end{figure}

% To define this process the FSM chart shown in Fig~\ref{fig_ANALYZE_FSM} can be implemented in the project\footnote{The chart was drawn in \url{creately.com}}.

% \begin{figure}
% \centering
% \includegraphics[width=11cm]{images/Analyze_FSM.png}\caption{The FSM chart for the analysis process.} \label{fig_ANALYZE_FSM}
% \end{figure}

% Now that the FSM is known, each state can be implemented. The pseudocodes of the states are defined in Tables~\ref{table_STATE1_IMP},~\ref{table_STATE2_IMP},~\ref{table_STATE3_IMP} and~\ref{table_STATE4_IMP}. These state functions have access to global variables which are defined in Table~\ref{table_GLOBAL_VARS}.

% \begin{table}
% \caption{Global variables of the FSM.}
% \center
% \label{table_GLOBAL_VARS}
% \begin{tabular}{|p{10cm}|}
% \hline
% \textbf{var} 
% targetPosition,
% initRobotPositions,
% initBallPosition,\\
% \quad oppGoalPosition:Position;

% \textbf{var}
% round, cnt:int;

% \textbf{var}
% successPositions,failPositions:Position[];

% \textbf{var}
% nextFunc2Run:function;\\

% \hline
% \end{tabular}
% \end{table}

% In Table~\ref{table_GLOBAL_VARS}, the \textit{targetPosition} is the location where the robot will perform its one-touch kick. This variable is initialized in every round. 

% The \textit{initRobotPositions} and \textit{initBallPosition} are the initial locations of the robots and the ball in every round. These variables are defined before the start of the test. \textit{oppGoalPosition} is the position of the center of the opponents goal line. This variable is defined before the start of the test.

% \textit{successPositions} and \textit{failPositions} are two vectors which store the positions according to their tags, \textbf{fail} or \textbf{success}.

% \textit{nextFunc2Run} is the function which will run in the next cycle (i.e. The next time a new vision dataset is received). This variable is defined as a C++ pointer to function in our executable code.


% \begin{table}[H]
% \caption{Implementation of the \textit{Setup} state.}
% \center
% \label{table_STATE1_IMP}
% \begin{tabular}{|p{10cm}|}
% \hline

% \textbf{function}
% $ S1\_setup$()\\
% \quad placeRobots(initRobotPositions);\\
% \quad placeBall(initBallPosition);\\
% \quad targetPosition = pickRandomPosition();\\
% \quad logData();\\
% \quad \textbf{if} round >= 500 \textbf{then}\\
% \quad\quad nextFunc2Run := S4\_done;\\
% \quad\quad cnt := 0;\\
% \quad \textbf{else if} cnt >= 130 \textbf{then}\\
% \quad\quad nextFunc2Run := S2\_initKick;\\
% \quad\quad cnt := 0;\\
% \quad \textbf{else}\\
% \quad\quad cnt := cnt + 1;\\

% \hline
% \end{tabular}
% \end{table}

% In \textit{S1\_setup}(), the robots and the ball have to get placed in the specified positions. This can be done by replacing the balls by hand or by robots and at last navigating the robots towards the specified positions.
% If the test is being performed in a simulator (e.g. grSim) it is possible to immediately place the robot by a command. Since the current test is performed in grSim, the robots will be placed using the placement commands implemented in grSim. These commands are shown as \textit{initRobotPositions}() and \textit{initBallPosition}() in the pseudocode. In every cycle (i.e. every time the function is called) A variable called \textit{cnt} is incremented by one. It is checked in an if statement to check whether it is time to transit to the next state or to stay in the current state. Another if statement checks if the round number has reached to 500, if so, the FSM will transit to the \textit{done} state which brings the process to an end.

% \begin{table}[H]
% \caption{Implementation of the \textit{Initial Kick} state.}
% \center
% \label{table_STATE2_IMP}
% \begin{tabular}{|p{10cm}|}
% \hline

% \textbf{function}
% $ S2\_initKick$()\\
% \quad circleKickBall(Robot3, targetPosition, 100);\\
% \quad ERRTNavigate2Point(Robot0, targetPosition);\\
% \quad logData();\\
% \quad \textbf{if} cnt >= 5 \textbf{then}\\
% \quad\quad nextFunc2Run := S3\_oneTouchKick;\\
% \quad\quad cnt := 0;\\
% \quad \textbf{else if} ballIsMoving() \textbf{then}\\
% \quad\quad cnt := cnt + 1;\\

% \hline
% \end{tabular}
% \end{table}

% In \textit{S2\_initKick}(), Robot \#3 tries to aim the \textit{targetPosition} and kick the ball towards it. Meanwhile, Robot \#0 will try to reach the \textit{targetPosition}. After a few moments when the ball is moving, the FSM will transit to the next state.


% \begin{table}[H]
% \caption{Implementation of the \textit{OneTouchKick} state.}
% \center
% \label{table_STATE3_IMP}
% \begin{tabular}{|p{10cm}|}
% \hline

% \textbf{function}
% $ S3\_oneTouchKick$()\\
% \quad halt(Robot3);\\
% \quad oneTouchDirect(Robot0, targetPosition, oppGoalPosition);\\
% \quad logData();\\
% \quad \textbf{if} ballIsOut() \textbf{then}\\
% \quad\quad \textbf{if} ballInGoal() \textbf{then}\\
% \quad\quad\quad successPositions.add(targetPosition);\\
% \quad\quad \textbf{else} \\ 
% \quad\quad\quad failPositions.add(targetPosition);\\
% \quad\quad nextFunc2Run := S1\_setup;\\
% \quad\quad round := round + 1;\\
% \quad \textbf{else if} ballIsNotMoving() \textbf{then}\\
% \quad\quad failPositions.add(targetPosition);\\
% \quad\quad nextFunc2Run := S1\_setup;\\
% \quad\quad round := round + 1;\\

% \hline
% \end{tabular}
% \end{table}

% In \textit{S3\_oneTouchKick}(), Robot \#3 gets into a halt mode and Robot \#0 will wait for the ball to reach it. Once the ball gets fetched by the robot. it will get kicked towards the \textit{oppGoalPosition}. The state transition will not happen until the ball exits the field or stops moving. When one of the transition conditions happen it will be judged whether the test was a success or a fail according to the position of the ball in relation with the goal.

% \begin{table}[H]
% \caption{Implementation of the \textit{Done} state.}
% \center
% \label{table_STATE4_IMP}
% \begin{tabular}{|p{10cm}|}
% \hline

% \textbf{function}
% $ S4\_done$()\\
% \quad halt(Robot3);\\
% \quad halt(Robot0);\\
% \quad logData();\\
% \quad logData(successPositions, GREEN);\\
% \quad logData(failPositions, RED);\\

% \hline
% \end{tabular}
% \end{table}

% In \textit{S4\_done}(), the robots are halted and the results are sent to the visualizer as red and green points.

% After running the code with 500 rounds in 1 hour and 20 minutes, the results were shown on the visualizer with 142 successful and 358 failed attempts. Fig~\ref{fig_ANALYZE_OUTPUT} shows the visualizer at the end of the test. It is now clearly seen which areas have a high possibility in scoring a goal by a one-touch kick under the tested conditions.

% \begin{figure}[H]
% \centering
% \includegraphics[width=10cm]{images/Analyze_output.png}\caption{The final results of the analysis.} \label{fig_ANALYZE_OUTPUT}
% \end{figure}

% Finally, it is worth to notice that the test has been performed in a simulation to give a better result in a short amount of time. It is clear that this test has to be made on robots in the real world. In that case, the number of rounds will obviously need to decrease to a few tens. The focus of this section was to show how an analysis procedure is taken in the Immortals AI project.

\newpage
\begin{thebibliography}{8}
\bibitem{ref_website}
Immortals Robotics Website, \url{http://www.immortals-robotics.com}.

\bibitem{ref_ETDP2020}
Immortals 2020 Extended Team Description Paper, \url{https://ssl.robocup.org/wp-content/uploads/2020/03/2020\_ETDP\_Immortals.pdf}.

\bibitem{ref_ETDP2019}
Immortals 2019 Team Description Paper, \url{https://ssl.robocup.org/wp-content/uploads/2019/03/2019\_ETDP\_Immortals.pdf}.

\bibitem{ref_ETDP2018}
Immortals 2018 Team Description Paper, \url{https://ssl.robocup.org/wp-content/uploads/2019/01/2018\_TDP\_Immortals.pdf}.

\bibitem{ref_github}
Immortals Open Source Project. \url{https://github.com/Immortals-Robotics}.

\bibitem{ref_grsim}
Monajjemi, Valiallah (Mani), Ali Koochakzadeh, and Saeed Shiry Ghidary. "grSim – RoboCup Small Size Robot Soccer Simulator." In Robot Soccer World Cup, pp. 450-460. Springer Berlin Heidelberg, 2011.

\bibitem{ref_cppbestpractices}
Turner, Jason. "C++ Best Practices"

\bibitem{ref_cppcoreguidelines}
Stroustrup, Bjarne and Sutter, Herb. "C++ Core Guidelines"

\bibitem{ref_toml}
TOML, A config file format for humans. \url{https://toml.io/}.

\bibitem{ref_json-schema}
JSON Schema, a declarative language that allows you to annotate and validate JSON documents. \url{https://json-schema.org/}.

% 3rd-party libraries
\bibitem{ref_3rd-party_gtest}
GoogleTest, Google Testing and Mocking Framework \url{https://github.com/google/googletest}

\bibitem{ref_3rd-party_vcpkg}
vcpkg, cross-platform C/C++ dependency manager from Microsoft \url{https://vcpkg.io/}

\bibitem{ref_3rd-party_asio}
Asio, cross-platform C++ library for network and low-level I/O programming \url{https://think-async.com/Asio/}

\bibitem{ref_3rd-party_quill}
Quill, asynchronous Low Latency C++ Logging Library \url{https://github.com/odygrd/quill}

\bibitem{ref_3rd-party_tomlplusplus}
toml++, header-only TOML config file parser and serializer for C++17. \url{https://github.com/marzer/tomlplusplus}

\bibitem{ref_3rd-party_eigen}
Eigen, C++ template library for linear algebra \url{https://eigen.tuxfamily.org/}

\bibitem{ref_3rd-party_homog2d}
homog2d, C++ 2D geometry library \url{https://github.com/skramm/homog2d}

\end{thebibliography}
\end{document}
